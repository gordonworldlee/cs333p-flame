
\documentclass[12pt]{article}



\usepackage{amssymb}
\usepackage{amsmath}
\usepackage{ifthen}
\usepackage[table]{xcolor}
\usepackage{minitoc}
\usepackage{array}
\usepackage{graphicx}

\definecolor{yellow}{cmyk}{0,0,1,0}
\renewcommand{\arraystretch}{1.4}
\newcommand{\R}{\mathbb{R}}

\usepackage{colortbl}
\usepackage[top=0.75in, bottom=0.75in, left=1.25in, right=1.25in]{geometry}

%% Page size
%\setlength{\oddsidemargin}{13pt}
%%\setlength{\evensidemargin}{-0.5in}
%\setlength{\textheight}{674pt}
%\setlength{\textwidth}{426pt}
%\setlength{\topmargin}{-10pt}
%\setlength{\voffset}{-30pt}

\renewcommand{\arraycolsep}{3pt}


\input{color_flatex}


% Page size
%\setlength{\oddsidemargin}{-0.25in}
%\setlength{\evensidemargin}{-0.25in}
%\setlength{\textheight}{9.25in}
%\setlength{\textwidth}{7in}
%\setlength{\topmargin}{-0.5in}


% ----------------------------------------------------------------
\title{Triangular Matrix Matrix Multiplication \\ CS 378 Programming for Correctness and Performance}

\author{Devangi N. Parikh}

\begin{document}
	\maketitle

\section{Operation}

Consider the operation
\[
B := L B 
\]
where $ L $ is a $ m \times m $ lower triangular matrix and $ B $ is a $ m \times n $ matrix.
This is a special case of  
matrix-matrix multiplication, 
with the {\sc l}ower triangular matrix on the {\sc l}eft, 
and the triangular matrix is {\sc n}ot transposed.
We will refer to this operation
as {\sc Trmm\_llnn} where the {\sc llnn} stands for
\underline{l}eft
\underline{l}ower
\underline{n}o-transpose
\underline{n}onunit diagonal.
The {\sc n}onunit diagonal means we will use the entries of the matrix that are stored on the diagonal.  In some other cases, the entries on the diagonal may implicitly be taked to all equal one ({\sc u}nit).

\section{Precondition and Postcondition}

In the precondition 
\[
B = \widehat B
\]
$ \widehat B $ denotes the original contents of $ B $.
This allows us to express the state upon completion, the postcondition, as
\[
B = L \widehat B.
\]
It is implicitly assumed that $ L $ is nonunit lower triangular.
\section{Partitioned Matrix Expressions and Loop Invariants}

There are two PMEs for this operation.

\subsection{PME 1}

To derive the second PME, partition
\[
L \rightarrow
\left(\begin{array}{c I c}
	L_{TL} & 0 \\ \whline
	L_{BL} & L_{BR}
\end{array}\right)
,
\quad \mbox{and} \quad
B \rightarrow 
\left(\begin{array}{c}
	B_T \\ \whline
	B_B
\end{array}\right).
\]
Substituting these into the postcondition
yields
\[
\left(\begin{array}{c}
	B_T \\ \whline
	B_B
\end{array}\right)
=
\left(\begin{array}{c I c}
	L_{TL} & 0 \\ \whline
	L_{BL} & L_{BR}
\end{array}\right)
\left(\begin{array}{c}
	\widehat B_T \\ \whline
	\widehat B_B
\end{array}\right)
\]
or, equivalently,
\[
\left(\begin{array}{c}
	B_T \\ \whline
	B_B
\end{array}\right)
=
\left(\begin{array}{c}
	L_{TL} \widehat B_T \\ \whline
	L_{BL} \widehat B_T + L_{BR} \widehat B_B
\end{array}\right),
\]
which we refer to as the first PME for this operations.

From this, we can choose two loop invariants:
\begin{description}
	\item
	{\bf Invariant 1:}
\begin{equation}
	\left(\begin{array}{c}
		B_T \\ \whline
		B_B
	\end{array}\right)
	= 
	\left(\begin{array}{c}
		\widehat B_T \\ \whline
		L_{BR} \widehat B_B
	\end{array}\right).	
\label{eq:inv1}
\end{equation}	
 \\
	(The top part has been left alone and the bottom part has been partially computed).
	\item
	{\bf Invariant 2:}
\begin{equation}
	\left(\begin{array}{c}
		B_T \\ \whline
		B_B
	\end{array}\right) = 
	\left(\begin{array}{c}
		\widehat B_T \\ \whline
		L_{BL} \widehat B_T + L_{BR} \widehat B_B
	\end{array}\right).
\label{eq:inv2}
	\end{equation} \\
	(The top part has been left alone and the bottom part has been completely computed).
\end{description}

\subsection{PME 2}

To derive the second PME, partition
\[
B \rightarrow \left(\begin{array}{c I c}
	B_L & B_R 
\end{array}\right)
\]
and do not partition $ L $.
Substituting these into the postcondition
yields
\[
\left(\begin{array}{c I c}
	B_L & B_R 
\end{array}\right)
=
L 
\left(\begin{array}{c I c}
	\widehat B_L & \widehat B_R 
\end{array}\right)
\]
or, equivalently,
\[
\left(\begin{array}{c I c}
	B_L & B_R 
\end{array}\right)
=
\left(\begin{array}{c I c}
	L \widehat B_L & L \widehat B_R 
\end{array}\right),
\]
which we refer to as the second PME.

From this, we can choose two more loop invariants:
\begin{description}
	\item
	{\bf Invariant 3:}

\begin{equation}
		\left(\begin{array}{c I c}
		B_L & B_R 
	\end{array}\right) =
	\left(\begin{array}{c I c}
		L \widehat B_L & \widehat B_R 
	\end{array}\right).
\label{eq:inv3}
\end{equation}
\\
	(The left part has been completely finished and the right part has been left untouched).
	\item
	{\bf Invariant 4:}
	
\begin{equation}
		\left(\begin{array}{c I c}
		B_L & B_R 
	\end{array}\right) =
	\left(\begin{array}{c I c}
		\widehat B_L & L \widehat B_R 
	\end{array}\right).
\label{eq:inv4}
\end{equation}\\
	(The left part has been completely finished and the right part has been left untouched).
\end{description}


\section{Deriving the Algorithms}

\subsection{Loop Invariant \eqref{eq:inv1}}

The unblocked algorithm for loop invariant \eqref{eq:inv1} is showing in Figure~\ref{fig:unb_inv1}. While the blocked algorithm for loop invariant \eqref{eq:inv1} is showing in Figure~\ref{fig:blk_inv1}.


\resetsteps



\resetsteps      % Reset all the commands to create a blank worksheet  

% Define the operation to be computed

\renewcommand{\operation}{ B = L B }

\renewcommand{\routinename}{\operation}

% Step 1a: Precondition 

\renewcommand{\precondition}{
	B = \widehat{B}
}

% Step 1b: Postcondition 

\renewcommand{\postcondition}{ 
	B = L \widehat B
}

% Step 2: Invariant 
% Note: Right-hand side of equalities must be updated appropriately

\renewcommand{\invariant}{
	\left(\begin{array}{c}
		B_T \\ \whline
		B_B 
	\end{array}\right)  = 
	\left(\begin{array}{c}
		\widehat B_T \\ \whline 
		L_{BR} \widehat B_B 
	\end{array}\right)
}

% Step 3: Loop-guard 

\renewcommand{\guard}{
	m( L_{BR} ) < m( L )
}

% Step 4: Initialize 

\renewcommand{\partitionings}{
	$
	L \rightarrow
	\left(\begin{array}{c I c} 
	L_{TL} & L_{TR} \\ \whline
	L_{BL} & L_{BR} 
	\end{array}\right) 
	$
	,
	$
	B \rightarrow
	\left(\begin{array}{c}
	B_{T} \\ \whline 
	B_{B}
	\end{array}\right)
	$
}

\renewcommand{\partitionsizes}{
	$ L_{BR} $ is $ 0 \times 0 $,
	$ B_B $ has $ 0 $ rows
}

% Step 5a: Repartition the operands 

\renewcommand{\repartitionings}{
	$  \left(\begin{array}{c I c}
	L_{TL} & L_{TR} \\ \whline 
	L_{BL} & L_{BR}
	\end{array}\right) 
	\rightarrow
	\left(\begin{array}{c c I c}
	L_{00} & l_{01} & L_{02} \\  
	l_{10}^T & \lambda_{11} & l_{12}^T \\ \whline 
	L_{20} & l_{21} & L_{22}
	\end{array}\right) 
	$
	,
	$  \left(\begin{array}{c}
	B_T \\ \whline
	B_B 
	\end{array}\right) 
	\rightarrow
	\left(\begin{array}{c}
	B_0 \\  
	b_1^T \\ \whline 
	B_2
	\end{array}\right)
	$
}

\renewcommand{\repartitionsizes}{
	$ \lambda_{11} $ is $ 1 \times 1 $,
	$ b_1 $ has $ 1 $ row}

% Step 5b: Move the double lines 

\renewcommand{\moveboundaries}{
	$  \left(\begin{array}{c I c}
	L_{TL} & L_{TR} \\ \whline 
	L_{BL} & L_{BR}
\end{array}\right) 
\rightarrow
\left(\begin{array}{c I c  c}
	L_{00} & l_{01} & L_{02} \\  \whline 
	l_{10}^T & \lambda_{11} & l_{12}^T \\ 
	L_{20} & l_{21} & L_{22}
\end{array}\right) 
$
,
$  \left(\begin{array}{c}
	B_T \\ \whline
	B_B 
\end{array}\right) 
\rightarrow
\left(\begin{array}{c}
	B_0 \\  \whline 
	b_1^T \\ 
	B_2
\end{array}\right)
$
}

% Step 6: State before update
% Note: The below needs editing consistent with loop-invariant!!!

\renewcommand{\beforeupdate}{$ 
	\left(\begin{array}{c}
	B_0 \\ 
	b_1^T \\  \whline 
	B_2
	\end{array}\right) 
	=
	\left(\begin{array}{c}
	\widehat B_0 \\
	\widehat b_1^T \\   \whline 
	L_{22} \widehat B_2
	\end{array}\right) 
$}


% Step 7: State after update
% Note: The below needs editing consistent with loop-invariant!!!

\renewcommand{\afterupdate}{$ 
		\left(\begin{array}{c}
		B_0 \\ \whline 
		b_1^T \\  
		B_2
		\end{array}\right) 
		=
		\left(\begin{array}{c}
		\widehat B_0 \\ \whline 
		\lambda_{11} \widehat b_1^T \\  
		l_{21} b_1^T + L_{22} \widehat B_2
		\end{array}\right) 
	$}


% Step 8: Insert the updates required to change the 
%         state from that given in Step 6 to that given in Step 7
% Note: The below needs editing!!!

\renewcommand{\update}{
	$
	\begin{array}{l}          % do not delete this line 
	B_2 := l_{21} b_1^T + B_2 \\
	b_1^T := \lambda_{11} b_1^T 
	\end{array}               % do not delete this line 
	$
}



\begin{figure}
	\begin{center}
		\FlaWorksheet
	\end{center}
	\caption{Unblocked Algorithm for Loop Invariant \eqref{eq:inv1}}
	\label{fig:unb_inv1}
\end{figure}


\resetsteps



\resetsteps      % Reset all the commands to create a blank worksheet  

% Define the operation to be computed

\renewcommand{\operation}{ B = L B }

\renewcommand{\routinename}{\operation}

% Step 1a: Precondition 

\renewcommand{\precondition}{
	B = \widehat{B}
}

% Step 1b: Postcondition 

\renewcommand{\postcondition}{ 
	B = L \widehat B
}

% Step 2: Invariant 
% Note: Right-hand side of equalities must be updated appropriately

\renewcommand{\invariant}{
	\left(\begin{array}{c}
		B_T \\ \whline
		B_B 
	\end{array}\right)  = 
	\left(\begin{array}{c}
		\widehat B_T \\ \whline 
		L_{BR} \widehat B_B 
	\end{array}\right)
}

% Step 3: Loop-guard 

\renewcommand{\guard}{
	m( L_{BR} ) < m( L )
}

% Step 4: Initialize 

\renewcommand{\partitionings}{
	$
	L \rightarrow
	\left(\begin{array}{c I c} 
	L_{TL} & L_{TR} \\ \whline
	L_{BL} & L_{BR} 
	\end{array}\right) 
	$
	,
	$
	B \rightarrow
	\left(\begin{array}{c}
	B_{T} \\ \whline 
	B_{B}
	\end{array}\right)
	$
}

\renewcommand{\partitionsizes}{
	$ L_{BR} $ is $ 0 \times 0 $,
	$ B_B $ has $ 0 $ rows
}

% Step 5a: Repartition the operands 

\renewcommand{\blocksize}{b}


\renewcommand{\repartitionings}{
	$  \left(\begin{array}{c I c}
	L_{TL} & L_{TR} \\ \whline 
	L_{BL} & L_{BR}
	\end{array}\right) 
	\rightarrow
	\left(\begin{array}{c c I c}
	L_{00} & L_{01} & L_{02} \\  
	L_{10} & L_{11} & L_{12} \\ \whline 
	L_{20} & L_{21} & L_{22}
	\end{array}\right) 
	$
	,
	$  \left(\begin{array}{c}
	B_T \\ \whline
	B_B 
	\end{array}\right) 
	\rightarrow
	\left(\begin{array}{c}
	B_0 \\  
	B_1 \\ \whline 
	B_2
	\end{array}\right)
	$
}

\renewcommand{\repartitionsizes}{
	$ L_{11} $ is $ b \times b $,
	$ B_1 $ has $ b $ rows}

% Step 5b: Move the double lines 

\renewcommand{\moveboundaries}{
	$  \left(\begin{array}{c I c}
	L_{TL} & L_{TR} \\ \whline 
	L_{BL} & L_{BR}
	\end{array}\right) 
	\leftarrow
	\left(\begin{array}{c I c  c}
	L_{00} & L_{01} & L_{02} \\   \whline 
	L_{10} & L_{11} & L_{12} \\
	L_{20} & L_{21} & L_{22}
\end{array}\right) 
	$
	,
	$  \left(\begin{array}{c}
	B_T \\ \whline
	B_B 
	\end{array}\right) 
	\leftarrow
	\left(\begin{array}{c}
	B_0 \\ \whline 
	B_1 \\  
	B_2
	\end{array}\right) 
	$
}

% Step 6: State before update
% Note: The below needs editing consistent with loop-invariant!!!

\renewcommand{\beforeupdate}{$ 
	\left(\begin{array}{c}
	B_0 \\
	B_1 \\  \whline 
	B_2
	\end{array}\right) 
	=
	\left(\begin{array}{c}
	\widehat B_0 \\ 
	\widehat B_1\\  \whline 
	L_{22} \widehat B_2
	\end{array}\right) 
$}


% Step 7: State after update
% Note: The below needs editing consistent with loop-invariant!!!

\renewcommand{\afterupdate}{$ 
		\left(\begin{array}{c}
		B_0 \\ \whline 
		B_1 \\  
		B_2
		\end{array}\right) 
		=
		\left(\begin{array}{c}
		\widehat B_0 \\ \whline 
	L_{11} \widehat B_1 \\  
		L_{21} B_1 + L_{22} \widehat B_2
		\end{array}\right) 
	$}


% Step 8: Insert the updates required to change the 
%         state from that given in Step 6 to that given in Step 7
% Note: The below needs editing!!!

\renewcommand{\update}{
	$
	\begin{array}{l}          % do not delete this line 
	B_2 := L_{21} B_1 + B_2 \\
	B_1 := L_{11} B_1 
	\end{array}               % do not delete this line 
	$
}



\begin{figure}
	\begin{center}
		\FlaWorksheet
	\end{center}
	\caption{Blocked Algorithm for Loop Invariant \eqref{eq:inv1}}
	\label{fig:blk_inv1}
\end{figure}

\section{Performance of the blocked and unblocked Algorithms}

Figure~\ref{fig:unbAlg1} shows the performance of the unblocked algorithm for loop invariant~\ref{eq:inv1} of TRMM.

\begin{figure}
\includegraphics[width=\textwidth]{FLAMEC/Plot_unb_GFLOPS.pdf}
\caption{Performance of the unblocked algorithm for loop invariant~\ref{eq:inv1} of TRMM}.
\label{fig:unbAlg1}
\end{figure}



\end{document}

