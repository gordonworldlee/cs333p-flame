\section{Operation}

Considering the operation
\[
B := B U
\]
where $ U $ is a $ m \times m $ upper triangular matrix and $ B $ is a $ n \times m $ matrix.
This is a special case of  
matrix-matrix multiplication, 
with the {\sc u}pper triangular matrix on the {\sc r}ight, 
and the triangular matrix is {\sc n}ot transposed.
We will refer to this operation
as {\sc Trmm\_runn} where the {\sc runn} stands for
\underline{r}ight
\underline{u}pper
\underline{n}o-transpose
\underline{n}onunit diagonal.
The {\sc n}onunit diagonal means we will use the entries of the matrix that are stored on the diagonal.

\section{Precondition and Post-condition}

In the precondition 
\[
B = \widehat B
\]
$ \widehat B $ denotes the original contents of $ B $. It is implicitly assumed that $ U $ is the upper triangular of nonunit units.
This allows us to express the state upon completion, the post-condition, as
\[
B = \widehat B U .
\]
\section{Partitioned Matrix Expressions and Loop Invariants}

There are two PMEs for this operation.

\subsection{PME 1}

To derive the first PME, partition
\[
U \rightarrow
\left(\begin{array}{c I c}
	U_{TL} & U_{TR} \\ \whline
	0 & U_{BR}
\end{array}\right)
,
\quad \mbox{and} \quad
B \rightarrow \left(\begin{array}{c I c}
	B_L & B_R 
\end{array}\right)
\]
Substituting these into the postcondition
yields
\[
\quad
\left(\begin{array}{c I c}
	B_L & B_R 
\end{array}\right)
=
\left(\begin{array}{c I c}
	\widehat B_L & \widehat B_R 
\end{array}\right)
\left(\begin{array}{c I c}
	U_{TL} & U_{TR} \\ \whline
	* & U_{BR}
\end{array}\right)
\]
or, equivalently,
\[
\left(\begin{array}{c I c}
	B_L & B_R 
\end{array}\right)
=
\left(\begin{array}{c I c}
	\widehat B_L U_{TL} & \widehat B_L U_{TR} + \widehat B_R U_{BR} 
\end{array}\right)
\]
which we refer to as the first PME for this operations.

From this, we can choose two loop invariants:
\begin{description}
	\item
	{\bf Invariant 1:}
\begin{equation}
	\left(\begin{array}{c I c}
	B_L & B_R 
        \end{array}\right)
	= 
	\left(\begin{array}{c I c}
	\widehat B_L & \widehat B_L U_{TR} + \widehat B_R U_{BR} 
        \end{array}\right).	
\label{eq:inv1}
\end{equation}	
 \\
	(The left part has been left alone and the right part has been fully computed).
	\item
	{\bf Invariant 2:}
\begin{equation}
	\left(\begin{array}{c I c}
	B_L & B_R 
        \end{array}\right) = 
	\left(\begin{array}{c I c}
	\widehat B_L & \widehat B_R U_{BR} 
        \end{array}\right).
\label{eq:inv2}
	\end{equation} \\
	(The left part has been left alone and the right part has been partially computed).
\end{description}

\subsection{PME 2}

To derive the second PME, partition
\[
B \rightarrow \left(\begin{array}{c}
		B_T \\ \whline
		B_B
	\end{array}\right)
\]
and do not partition $ U $.
Substituting these into the postcondition
yields
\[
\left(\begin{array}{c}
    B_T \\ \whline
    B_B
\end{array}\right)
=
\left(\begin{array}{c}
    \widehat B_T \\ \whline
    \widehat B_B
\end{array}\right) 
U
\]
or, equivalently,
\[
\left(\begin{array}{c}
    B_T \\ \whline
    B_B
\end{array}\right)
=
\left(\begin{array}{c}
    \widehat B_T U \\ \whline
    \widehat B_B U
\end{array}\right) 
\]
which we refer to as the second PME.

From this, we can choose two more loop invariants:
\begin{description}
	\item
	{\bf Invariant 3:}

\begin{equation}
\left(\begin{array}{c}
    B_T \\ \whline
    B_B
\end{array}\right)
=
\left(\begin{array}{c}
    \widehat B_T U \\ \whline
    \widehat B_B
\end{array}\right) .
\label{eq:inv3}
\end{equation}
\\
	(The top part has been completely finished and the bottom part has been left untouched).
	\item
	{\bf Invariant 4:}
	
\begin{equation}
\left(\begin{array}{c}
    B_T \\ \whline
    B_B
\end{array}\right)
=
\left(\begin{array}{c}
    \widehat B_T  \\ \whline
    \widehat B_B U
\end{array}\right) ..
\label{eq:inv4}
\end{equation}\\
	(The bottom part has been completely finished and the top part has been left untouched).
\end{description}


\section{Deriving the Algorithms}

\subsection{Loop Invariant 1}

The unblocked algorithm Trmm\_runn\_unb\_var1 for loop invariant 1 is showing in Figure~\ref{fig:unb_inv1}. While the blocked algorithm Trmm\_runn\_blk\_var1 for loop invariant 1 is showing in Figure~\ref{fig:blk_inv1}.


\resetsteps

\input{Derivations/trmm_runn_unb_var1}

\begin{figure}
	\begin{center}
		\FlaWorksheet
	\end{center}
	\caption{Unblocked Algorithm for Loop Invariant 1}
	\label{fig:unb_inv1}
\end{figure}


\resetsteps


\resetsteps      % Reset all the commands to create a blank worksheet  

% Define the operation to be computed

\renewcommand{\operation}{ \left[ B \right] := \mbox{\sc Trmm\_runn\_blk\_var1}( B, U ) }

\renewcommand{\routinename}{\operation}

% Step 1a: Precondition 

\renewcommand{\precondition}{
  B = \widehat{B}
}

% Step 1b: Postcondition 

\renewcommand{\postcondition}{ 
  \left[B \right]
  =
  \mbox{Trmm\_runn}( \widehat{B}, U )
}

% Step 2: Invariant 
% Note: Right-hand side of equalities must be updated appropriately

\renewcommand{\invariant}{
  \left(\begin{array}{c I c}
     B_L & B_R
  \end{array}\right)  = 
  \left(\begin{array}{c I c}
	\widehat B_L & \widehat B_L U_{TR} + \widehat B_R U_{BR} 
    \end{array}\right)
}

% Step 3: Loop-guard 

\renewcommand{\guard}{
  n( B_R ) < n( B )
}

% Step 4: Initialize 

\renewcommand{\partitionings}{
  $
  B \rightarrow
  \left(\begin{array}{c I c}
     B_L & B_R
  \end{array}\right)
  $
,
  $
  U \rightarrow
  \left(\begin{array}{c I c} 
     U_{TL} & U_{TR} \\ \whline
     U_{BL} & U_{BR} 
  \end{array}\right) 
  $
}

\renewcommand{\partitionsizes}{
  $ B_R $ has $ 0 $ columns,
  $ U_{BR} $ is $ 0 \times 0 $
}

% Step 5a: Repartition the operands 

\renewcommand{\blocksize}{b}

\renewcommand{\repartitionings}{
  $  \left(\begin{array}{c I c}
     B_L & B_R
  \end{array}\right)
  \rightarrow
  \left(\begin{array}{c c I c}
     B_0 & B_1 & B_2
  \end{array}\right)
  $
,
  $  \left(\begin{array}{c I c}
     U_{TL} & U_{TR} \\ \whline 
     0 & U_{BR}
  \end{array}\right) 
  \rightarrow
  \left(\begin{array}{c c I c}
     U_{00} & U_{01} & U_{02} \\  
     0 & U_{11} & U_{12} \\ \whline 
     0 & 0 & U_{22}
  \end{array}\right) 
  $
}

\renewcommand{\repartitionsizes}{
  $ B_1 $ has $ b $ columns,
  $ U_{11} $ is $ b \times b $}

% Step 5b: Move the double lines 

\renewcommand{\moveboundaries}{
$  
  \left(\begin{array}{c I c}
     B_L & B_R
  \end{array}\right)
  \leftarrow
  \left(\begin{array}{c I c c}
     B_0 & B_1 & B_2
  \end{array}\right)
  $
,
$  \left(\begin{array}{c I c}
     U_{TL} & U_{TR} \\ \whline 
     0 & U_{BR}
  \end{array}\right) 
  \leftarrow
  \left(\begin{array}{c I c c}
     U_{00} & U_{01} & U_{02} \\ \whline 
     0 & U_{11} & U_{12} \\  
     0 & 0 & U_{22}
  \end{array}\right) 
  $
}

% Step 6: State before update
% Note: The below needs editing consistent with loop-invariant!!!

\renewcommand{\beforeupdate}{
$  \left(\begin{array}{c c c}
     B_0 & B_1 & B_2
  \end{array}\right)  = 
  \left(\begin{array}{c c c}
     \widehat B_0 & \widehat B_1 & \widehat B_0 U_{02} + \widehat B_1 U_{12}^T + \widehat B_2 U_{22} 
  \end{array}\right)
$
}


% Step 7: State after update

\renewcommand{\afterupdate}{
$ \left(\begin{array}{c c c}
     B_0 & B_1 & B_2
  \end{array}\right)  = 
  \left(\begin{array}{c c c}
	\widehat B_0 & \widehat B_0 U_{01} + \widehat B_1 U_{11} & \widehat B_0 U_{02} + \widehat B_1 U_{12}^T + \widehat B_2 U_{22}
    \end{array}\right)
$
}

% Step 8: Insert the updates required to change the 
%         state from that given in Step 6 to that given in Step 7

\renewcommand{\update}{
$
  \begin{array}{l}
    B_1 = B_1 U_{11} \\
    B_1 = B_1 + B_0 U_{01} \\
  \end{array}               % do not delete this line 
$
}




\begin{figure}
	\begin{center}
		\FlaWorksheet
	\end{center}
	\caption{Blocked Algorithm for Loop Invariant 1}
	\label{fig:blk_inv1}
\end{figure}