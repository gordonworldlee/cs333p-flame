
\resetsteps      % Reset all the commands to create a blank worksheet  

% Define the operation to be computed

\renewcommand{\operation}{ \left[ C \right] := \mbox{\sc Syrk\_ln\_blk\_var3}( A, C ) }

\renewcommand{\routinename}{\operation}

% Step 1a: Precondition 

\renewcommand{\precondition}{
  C = \widehat{C}
}

% Step 1b: Postcondition 

\renewcommand{\postcondition}{ 
  \left[C \right]
  =
  \mbox{Syrk\_ln}( A, \widehat{C} )
}

% Step 2: Invariant 
% Note: Right-hand side of equalities must be updated appropriately

\renewcommand{\invariant}{
  \left(\begin{array}{c I c}
     C_{TL} & C_{TR} \\ \whline 
     C_{BL} & C_{BR}
  \end{array}\right) =
  \left(\begin{array}{c I c}
    \widehat C_{TL} & * \\ \whline
    \widehat C_{BL} & A_{B}A_{B}^{T} + \widehat C_{BR}
  \end{array}\right)
}

% Step 3: Loop-guard 

\renewcommand{\guard}{
  m( A_B ) < m( A )
}

% Step 4: Initialize 

\renewcommand{\partitionings}{
  $
  A \rightarrow
  \left(\begin{array}{c}
     A_{T} \\ \whline 
     A_{B}
  \end{array}\right)
  $
,
  $
  C \rightarrow
  \left(\begin{array}{c I c} 
     C_{TL} & C_{TR} \\ \whline
     C_{BL} & C_{BR} 
  \end{array}\right) 
  $
}

\renewcommand{\partitionsizes}{
  $ A_B $ has $ 0 $ rows,
  $ C_{BR} $ is $ 0 \times 0 $
}

% Step 5a: Repartition the operands 

\renewcommand{\blocksize}{b}

\renewcommand{\repartitionings}{
  $  \left(\begin{array}{c}
     A_T \\ \whline
     A_B 
  \end{array}\right) 
  \rightarrow
  \left(\begin{array}{c}
     A_0 \\  
     A_1 \\ \whline 
     A_2
  \end{array}\right)
  $
,
  $  \left(\begin{array}{c I c}
     C_{TL} & C_{TR} \\ \whline 
     C_{BL} & C_{BR}
  \end{array}\right) 
  \rightarrow
  \left(\begin{array}{c c I c}
     C_{00} & C_{01} & C_{02} \\  
     C_{10} & C_{11} & C_{12} \\ \whline 
     C_{20} & C_{21} & C_{22}
  \end{array}\right) 
  $
}

\renewcommand{\repartitionsizes}{
  $ A_1 $ has $ b $ rows,
  $ C_{11} $ is $ b \times b $}

% Step 5b: Move the double lines 

\renewcommand{\moveboundaries}{
$  \left(\begin{array}{c}
     A_T \\ \whline
     A_B 
  \end{array}\right) 
  \leftarrow
  \left(\begin{array}{c}
     A_0 \\ \whline 
     A_1 \\  
     A_2
  \end{array}\right) 
  $
,
$  \left(\begin{array}{c I c}
     C_{TL} & C_{TR} \\ \whline 
     C_{BL} & C_{BR}
  \end{array}\right) 
  \leftarrow
  \left(\begin{array}{c I c c}
     C_{00} & C_{01} & C_{02} \\ \whline 
     C_{10} & C_{11} & C_{12} \\  
     C_{20} & C_{21} & C_{22}
  \end{array}\right) 
  $
}

% Step 6: State before update
% Note: The below needs editing consistent with loop-invariant!!!

\renewcommand{\beforeupdate}{$
\left(\begin{array}{c c c}
     C_{00} & * & * \\  
     C_{10}^T & C_{11} & * \\ 
     C_{20} & C_{21} & C_{22}
  \end{array}\right)  =
  \left(\begin{array}{c c c}
     \widehat C_{00} & \ * & * \\  
     \widehat C_{10}^T & \ \widehat C_{11} & * \\ 
     \widehat C_{20} & \widehat C_{21} & A_2 A_2^T + \widehat C_{22}
  \end{array}\right) 
$}


% Step 7: State after update
% Note: The below needs editing consistent with loop-invariant!!!

\renewcommand{\afterupdate}{$ 
\left(\begin{array}{c c c}
     C_{00} & * & * \\  
     C_{10}^T & C_{11} & * \\ 
     C_{20} & C_{21} & C_{22}
  \end{array}\right)  =
  \left(\begin{array}{c c c}
     \widehat C_{00} & * & * \\  
     \widehat C_{10}^T & \ A_1 A_1^T + \widehat C_{11} & * \\ 
     \widehat C_{20} & \ A_2 A_1^T + \widehat C_{21} & A_2 A_2^T + \widehat C_{22}
  \end{array}\right) 
$
}


% Step 8: Insert the updates required to change the 
%         state from that given in Step 6 to that given in Step 7

\renewcommand{\update}{
$
  \begin{array}{l}          % do not delete this line 
    C_{11} = A_1 A_1^T + C_{11} \\
    C_{21} = A_2 A_1^T + C_{21} \\
  \end{array}               % do not delete this line 
$
}


