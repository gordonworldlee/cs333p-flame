
\resetsteps      % Reset all the commands to create a blank worksheet  

% Define the operation to be computed

\renewcommand{\operation}{ \left[ B \right] := \mbox{\sc Trmm\_runn\_unb\_var1}( B, U ) }

\renewcommand{\routinename}{\operation}

% Step 1a: Precondition 

\renewcommand{\precondition}{
  B = \widehat{B}
}

% Step 1b: Postcondition 

\renewcommand{\postcondition}{ 
  \left[B \right]
  =
  \mbox{Trmm\_runn}( \widehat{B}, U )
}

% Step 2: Invariant 
% Note: Right-hand side of equalities must be updated appropriately

\renewcommand{\invariant}{
  \left(\begin{array}{c I c}
     B_L & B_R
  \end{array}\right)  = 
  \left(\begin{array}{c I c}
	\widehat B_L & \widehat B_L U_{TR} + \widehat B_R U_{BR} 
    \end{array}\right)
}

% Step 3: Loop-guard 

\renewcommand{\guard}{
  n( B_R ) < n( B )
}

% Step 4: Initialize 

\renewcommand{\partitionings}{
  $
  B \rightarrow
  \left(\begin{array}{c I c}
     B_L & B_R
  \end{array}\right)
  $
,
  $
  U \rightarrow
  \left(\begin{array}{c I c} 
     U_{TL} & U_{TR} \\ \whline
     U_{BL} & U_{BR} 
  \end{array}\right) 
  $
}

\renewcommand{\partitionsizes}{
  $ B_R $ has $ 0 $ columns,
  $ U_{BR} $ is $ 0 \times 0 $
}

% Step 5a: Repartition the operands 

\renewcommand{\repartitionings}{
  $  \left(\begin{array}{c I c}
     B_L & B_R
  \end{array}\right)
  \rightarrow
  \left(\begin{array}{c c I c}
     B_0 & b_1 & B_2
  \end{array}\right)
  $
,
  $  \left(\begin{array}{c I c}
     U_{TL} & U_{TR} \\ \whline 
     0 & U_{BR}
  \end{array}\right) 
  \rightarrow
  \left(\begin{array}{c c I c}
     U_{00} & u_{01} & U_{02} \\  
     0 & \upsilon_{11} & u_{12}^T \\ \whline 
     0 & 0 & U_{22}
  \end{array}\right) 
  $
}

\renewcommand{\repartitionsizes}{
  $ b_1 $ has $ 1 $ column,
  $ \upsilon_{11} $ is $ 1 \times 1 $}

% Step 5b: Move the double lines 

\renewcommand{\moveboundaries}{
$  
  \left(\begin{array}{c I c}
     B_L & B_R
  \end{array}\right)
  \leftarrow
  \left(\begin{array}{c I c c}
     B_0 & b_1 & B_2
  \end{array}\right)
  $
,
$  \left(\begin{array}{c I c}
     U_{TL} & U_{TR} \\ \whline 
     0 & U_{BR}
  \end{array}\right) 
  \leftarrow
  \left(\begin{array}{c I c c}
     U_{00} & u_{01} & U_{02} \\ \whline 
     0 & \upsilon_{11} & u_{12}^T \\  
     0 & 0 & U_{22}
  \end{array}\right) 
  $
}

% Step 6: State before update
% Note: The below needs editing consistent with loop-invariant!!!

\renewcommand{\beforeupdate}{
$  \left(\begin{array}{c c c}
     B_0 & b_1 & B_2
  \end{array}\right)  = 
  \left(\begin{array}{c c c}
     \widehat B_0 & \widehat b_1 & \widehat B_0 U_{02} + \widehat b_1 u_{12}^T + \widehat B_2 u_{22} 
  \end{array}\right)
$
}


% Step 7: State after update
% Note: The below needs editing consistent with loop-invariant!!!

\renewcommand{\afterupdate}{
$ \left(\begin{array}{c c c}
     B_0 & b_1 & B_2
  \end{array}\right)  = 
  \left(\begin{array}{c c c}
	\widehat B_0 & \widehat B_0 u_{01} + \widehat b_1 u_{11} & \widehat B_0 U_{02} + \widehat b_1 u_{12}^T + \widehat B_2 U_{22}
    \end{array}\right)
$
}

% Step 8: Insert the updates required to change the 
%         state from that given in Step 6 to that given in Step 7
% Note: The below needs editing!!!

\renewcommand{\update}{
$
  \begin{array}{l}
    b_1 = b_1 u_{11} \\
    b_1 = b_1 + B_0 u_{01} \\
  \end{array}               % do not delete this line 
$
}


