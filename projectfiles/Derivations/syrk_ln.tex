\section{Operation}

Considering the operation
\[
C := AA^T + C
\]
where $ A $ is a $ n \times m $ matrix and $ C $ is a $ n \times n $ symmetric matrix stored in the lower triangle part.
This is a case of symmetric rank k update, 
with the {\sc S}ymmetric matrix on the {\sc r}ight, 
and the matrix is {\sc n}ot transposed.
We will refer to this operation
as {\sc Syrk\_ln} where the {\sc ln} stands for
\underline{l}ower
\underline{n}onunit diagonal.
The {\sc n}onunit diagonal means we will use the entries of the matrix that are stored on the diagonal.

\section{Precondition and Post-condition}

In the precondition 
\[
C = \widehat C
\]
$ \widehat C $ denotes the original contents of the symmetric matrix $ C $ stored in the lower triangular part. Matrix $ A $ is a regular $ n \times m $ matrix.
This allows us to express the state upon completion, the post-condition, as
\[
C = AA^T + \widehat C .
\]

\section{Partitioned Matrix Expressions and Loop Invariants}

There are two PMEs for this operation.

\subsection{PME 1}

To derive the first PME, partition
\[
A \;\longrightarrow\;
\left(\begin{array}{c I c}
  A_{L} & A_{R}
\end{array}\right).
\]
and do not partition C.
Substituting this into the post-condition
yields
\[
C \;=\;
\left(\begin{array}{c I c}
  A_{L} & A_{R}
\end{array}\right)
\left(\begin{array}{c I c}
  A_{L} & A_{R}
\end{array}\right)^{T}
\;+\;
\widehat C,
\]
or, equivalently,
\[
C \;=\; A_{L}A_{L}^{T} \;+\; A_{R}A_{R}^{T} \;+\; \widehat C,
\]
which we refer to as the first PME for this operation. From this, we can choose two loop invariants:
\begin{description}
  \item
  {\bf Invariant 1:}
\begin{equation}
    C
  \;=\;
    A_{L}A_{L}^{T} + \widehat C
\label{eq:inv1}
\end{equation}
\\
(The left block has been fully computed; the right block remains as in \(\widehat C\).)

  \item
  {\bf Invariant 2:}
\begin{equation}
    C
  \;=\;
    A_{R}A_{R}^{T} + \widehat C
\label{eq:inv2}
\end{equation}
\\
(The right block has been fully computed; the left block remains as in \(\widehat C\).)
\end{description}

\subsection{PME 2}

To derive the second PME, partition
\[
A \;\longrightarrow\;
\left(\begin{array}{c}
  A_{T} \\ \whline
  A_{B}
\end{array}\right),
\quad
C \;\longrightarrow\;
\left(\begin{array}{c I c}
  C_{TL} & * \\ \whline
  C_{BL} & C_{BR}
\end{array}\right).
\]
Substituting these into the post-condition yields
\[
\left(\begin{array}{c I c}
  C_{TL} & * \\ \whline
  C_{BL} & C_{BR}
\end{array}\right)
=
\left(\begin{array}{c}
  A_{T} \\ \whline
  A_{B}
\end{array}\right)
\left(\begin{array}{c}
  A_{T} \\ \whline
  A_{B}
\end{array}\right)^{T}
\;+\;
\left(\begin{array}{c I c}
  \widehat C_{TL} & * \\ \whline
  \widehat C_{BL} & \widehat C_{BR}
\end{array}\right),
\]
or, equivalently,
\[
\left(\begin{array}{c I c}
  \widehat C_{TL} + A_{T}A_{T}^{T} & * \\ \whline
  \widehat C_{BL} + A_{B}A_{T}^{T} & \widehat C_{BR} + A_{B}A_{B}^{T}
\end{array}\right),
\]
which we refer to as the second PME for this operation. From this, we can choose four loop invariants:
\begin{description}
  \item
  {\bf Invariant 1:}
  \begin{equation}
  \left(\begin{array}{c I c}
    C_{TL} & C_{TR} \\ \whline
    C_{BL} & C_{BR}
  \end{array}\right)
  =
  \left(\begin{array}{c I c}
    A_{T}A_{T}^{T} + \widehat C_{TL} & * \\ \whline
    \widehat C_{BL} + A_{B}A_{T}^{T} & \widehat C_{BR}
  \end{array}\right)
  \label{eq:inv1}
  \end{equation}
  (The first column has been fully computed; the second column remains as in \(\widehat C\).)

  \item
  {\bf Invariant 2:}
  \begin{equation}
  \left(\begin{array}{c I c}
    C_{TL} & C_{TR} \\ \whline
    C_{BL} & C_{BR}
  \end{array}\right)
  =
  \left(\begin{array}{c I c}
    \widehat C_{TL} & * \\ \whline
    \widehat C_{BL} + A_{B}A_{T}^{T} & A_{B}A_{B}^{T} + \widehat C_{BR}
  \end{array}\right)
  \label{eq:inv2}
  \end{equation}
  (The bottom row has been fully computed; the top row remains as in \(\widehat C\).)

  \item
  {\bf Invariant 3:}
  \begin{equation}
  \left(\begin{array}{c I c}
    C_{TL} & C_{TR} \\ \whline
    C_{BL} & C_{BR}
  \end{array}\right)
  =
  \left(\begin{array}{c I c}
    \widehat C_{TL} & * \\ \whline
    \widehat C_{BL} & A_{B}A_{B}^{T} + \widehat C_{BR}
  \end{array}\right)
  \label{eq:inv3}
  \end{equation}
  (Only the bottom-right block has been fully computed; all other blocks remain as in \(\widehat C\).)

  \item
  {\bf Invariant 4:}
  \begin{equation}
  \left(\begin{array}{c I c}
    C_{TL} & C_{TR} \\ \whline
    C_{BL} & C_{BR}
  \end{array}\right)
  =
  \left(\begin{array}{c I c}
    A_{T}A_{T}^{T} + \widehat C_{TL} & * \\ \whline
    \widehat C_{BL} & \widehat C_{BR}
  \end{array}\right)
  \label{eq:inv4}
  \end{equation}
  (Only the top-left block has been fully computed; all other blocks remain as in \(\widehat C\).)
\end{description}


\section{Deriving the Algorithms}

\subsection{Loop Invariant 3}

The unblocked algorithm Syrk\_ln\_unb\_var3(A, C) for loop invariant 3 is showing in Figure~\ref{fig:unb_inv3}. While the blocked algorithm Syrk\_ln\_blk\_var3(A, C) for loop invariant 3 is seen in Figure~\ref{fig:blk_inv3}. \\


\resetsteps

\resetsteps      % Reset all the commands to create a blank worksheet  

% Define the operation to be computed

\renewcommand{\operation}{ \left[ C \right] := \mbox{\sc Syrk\_ln\_unb\_var3}( A, C ) }

\renewcommand{\routinename}{\operation}

% Step 1a: Precondition 

\renewcommand{\precondition}{
  C = \widehat{C}
}

% Step 1b: Postcondition 

\renewcommand{\postcondition}{ 
  \left[C \right]
  =
  \mbox{Syrk\_ln}( A, \widehat{C} )
}

% Step 2: Invariant 
% Note: Right-hand side of equalities must be updated appropriately

\renewcommand{\invariant}{
  \left(\begin{array}{c I c}
     C_{TL} & C_{TR} \\ \whline 
     C_{BL} & C_{BR}
  \end{array}\right) =
  \left(\begin{array}{c I c}
    \widehat C_{TL} & * \\ \whline
    \widehat C_{BL} & A_{B}A_{B}^{T} + \widehat C_{BR}
  \end{array}\right)
}

% Step 3: Loop-guard 

\renewcommand{\guard}{
  m( A_B ) < m( A )
}

% Step 4: Initialize 

\renewcommand{\partitionings}{
  $
  A \rightarrow
  \left(\begin{array}{c}
     A_{T} \\ \whline 
     A_{B}
  \end{array}\right)
  $
,
  $
  C \rightarrow
  \left(\begin{array}{c I c} 
     C_{TL} & C_{TR} \\ \whline
     C_{BL} & C_{BR} 
  \end{array}\right) 
  $
}

\renewcommand{\partitionsizes}{
  $ A_B $ has $ 0 $ rows,
  $ C_{BR} $ is $ 0 \times 0 $
}

% Step 5a: Repartition the operands 

\renewcommand{\repartitionings}{
  $  \left(\begin{array}{c}
     A_T \\ \whline
     A_B 
  \end{array}\right) 
  \rightarrow
  \left(\begin{array}{c}
     A_0 \\  
     a_1^T \\ \whline 
     A_2
  \end{array}\right)
  $
,
  $  \left(\begin{array}{c I c}
     C_{TL} & C_{TR} \\ \whline 
     C_{BL} & C_{BR}
  \end{array}\right) 
  \rightarrow
  \left(\begin{array}{c c I c}
     C_{00} & c_{01} & C_{02} \\  
     c_{10}^T & \gamma_{11} & c_{12}^T \\ \whline 
     C_{20} & c_{21} & C_{22}
  \end{array}\right) 
  $
}

\renewcommand{\repartitionsizes}{
  $ a_1 $ has $ 1 $ row,
  $ \gamma_{11} $ is $ 1 \times 1 $}

% Step 5b: Move the double lines 

\renewcommand{\moveboundaries}{
$  \left(\begin{array}{c}
     A_T \\ \whline
     A_B 
  \end{array}\right) 
  \leftarrow
  \left(\begin{array}{c}
     A_0 \\ \whline 
     a_1^T \\  
     A_2
  \end{array}\right) 
  $
,
$  \left(\begin{array}{c I c}
     C_{TL} & C_{TR} \\ \whline 
     C_{BL} & C_{BR}
  \end{array}\right) 
  \leftarrow
  \left(\begin{array}{c I c c}
     C_{00} & c_{01} & C_{02} \\ \whline 
     c_{10}^T & \gamma_{11} & c_{12}^T \\  
     C_{20} & c_{21} & C_{22}
  \end{array}\right) 
  $
}

% Step 6: State before update
% Note: The below needs editing consistent with loop-invariant!!!

\renewcommand{\beforeupdate}{$
\left(\begin{array}{c c c}
     C_{00} & * & * \\  
     c_{10}^T & \gamma_{11} & * \\ 
     C_{20} & c_{21} & C_{22}
  \end{array}\right)  =
  \left(\begin{array}{c c c}
     \widehat C_{00} & * & * \\  
     \widehat c_{10}^T & \widehat \gamma_{11} & * \\ 
     \widehat C_{20} & \widehat c_{21} & A_2 A_2^T + \widehat C_{22}
  \end{array}\right) 
$}


% Step 7: State after update
% Note: The below needs editing consistent with loop-invariant!!!

\renewcommand{\afterupdate}{$ 
\left(\begin{array}{c c c}
     C_{00} & * & * \\  
     c_{10}^T & \gamma_{11} & * \\ 
     C_{20} & c_{21} & C_{22}
  \end{array}\right)  =
  \left(\begin{array}{c c c}
     \widehat C_{00} & * & * \\  
     \widehat c_{10}^T & a_1^T a_1^T^T + \widehat \gamma_{11} & * \\ 
     \widehat C_{20} & A_2 a_1^T^T + \widehat c_{21} & A_2 A_2^T + \widehat C_{22}
  \end{array}\right) 
$
}


% Step 8: Insert the updates required to change the 
%         state from that given in Step 6 to that given in Step 7

\renewcommand{\update}{
$
  \begin{array}{l}          % do not delete this line 
    \gamma_{11} = a_1^T a_1^T^T + \gamma_{11} \\
    c_{21} = A_2 a_1^T^T + c_{21} \\
  \end{array}               % do not delete this line 
$
}




\begin{figure}
	\begin{center}
		\FlaWorksheet
	\end{center}
	\caption{Unblocked Algorithm for Loop Invariant 3}
	\label{fig:unb_inv3}
\end{figure}


\resetsteps


\resetsteps      % Reset all the commands to create a blank worksheet  

% Define the operation to be computed

\renewcommand{\operation}{ \left[ C \right] := \mbox{\sc Syrk\_ln\_blk\_var3}( A, C ) }

\renewcommand{\routinename}{\operation}

% Step 1a: Precondition 

\renewcommand{\precondition}{
  C = \widehat{C}
}

% Step 1b: Postcondition 

\renewcommand{\postcondition}{ 
  \left[C \right]
  =
  \mbox{Syrk\_ln}( A, \widehat{C} )
}

% Step 2: Invariant 
% Note: Right-hand side of equalities must be updated appropriately

\renewcommand{\invariant}{
  \left(\begin{array}{c I c}
     C_{TL} & C_{TR} \\ \whline 
     C_{BL} & C_{BR}
  \end{array}\right) =
  \left(\begin{array}{c I c}
    \widehat C_{TL} & * \\ \whline
    \widehat C_{BL} & A_{B}A_{B}^{T} + \widehat C_{BR}
  \end{array}\right)
}

% Step 3: Loop-guard 

\renewcommand{\guard}{
  m( A_B ) < m( A )
}

% Step 4: Initialize 

\renewcommand{\partitionings}{
  $
  A \rightarrow
  \left(\begin{array}{c}
     A_{T} \\ \whline 
     A_{B}
  \end{array}\right)
  $
,
  $
  C \rightarrow
  \left(\begin{array}{c I c} 
     C_{TL} & C_{TR} \\ \whline
     C_{BL} & C_{BR} 
  \end{array}\right) 
  $
}

\renewcommand{\partitionsizes}{
  $ A_B $ has $ 0 $ rows,
  $ C_{BR} $ is $ 0 \times 0 $
}

% Step 5a: Repartition the operands 

\renewcommand{\blocksize}{b}

\renewcommand{\repartitionings}{
  $  \left(\begin{array}{c}
     A_T \\ \whline
     A_B 
  \end{array}\right) 
  \rightarrow
  \left(\begin{array}{c}
     A_0 \\  
     A_1 \\ \whline 
     A_2
  \end{array}\right)
  $
,
  $  \left(\begin{array}{c I c}
     C_{TL} & C_{TR} \\ \whline 
     C_{BL} & C_{BR}
  \end{array}\right) 
  \rightarrow
  \left(\begin{array}{c c I c}
     C_{00} & C_{01} & C_{02} \\  
     C_{10} & C_{11} & C_{12} \\ \whline 
     C_{20} & C_{21} & C_{22}
  \end{array}\right) 
  $
}

\renewcommand{\repartitionsizes}{
  $ A_1 $ has $ b $ rows,
  $ C_{11} $ is $ b \times b $}

% Step 5b: Move the double lines 

\renewcommand{\moveboundaries}{
$  \left(\begin{array}{c}
     A_T \\ \whline
     A_B 
  \end{array}\right) 
  \leftarrow
  \left(\begin{array}{c}
     A_0 \\ \whline 
     A_1 \\  
     A_2
  \end{array}\right) 
  $
,
$  \left(\begin{array}{c I c}
     C_{TL} & C_{TR} \\ \whline 
     C_{BL} & C_{BR}
  \end{array}\right) 
  \leftarrow
  \left(\begin{array}{c I c c}
     C_{00} & C_{01} & C_{02} \\ \whline 
     C_{10} & C_{11} & C_{12} \\  
     C_{20} & C_{21} & C_{22}
  \end{array}\right) 
  $
}

% Step 6: State before update
% Note: The below needs editing consistent with loop-invariant!!!

\renewcommand{\beforeupdate}{$
\left(\begin{array}{c c c}
     C_{00} & * & * \\  
     C_{10}^T & C_{11} & * \\ 
     C_{20} & C_{21} & C_{22}
  \end{array}\right)  =
  \left(\begin{array}{c c c}
     \widehat C_{00} & \ * & * \\  
     \widehat C_{10}^T & \ \widehat C_{11} & * \\ 
     \widehat C_{20} & \widehat C_{21} & A_2 A_2^T + \widehat C_{22}
  \end{array}\right) 
$}


% Step 7: State after update
% Note: The below needs editing consistent with loop-invariant!!!

\renewcommand{\afterupdate}{$ 
\left(\begin{array}{c c c}
     C_{00} & * & * \\  
     C_{10}^T & C_{11} & * \\ 
     C_{20} & C_{21} & C_{22}
  \end{array}\right)  =
  \left(\begin{array}{c c c}
     \widehat C_{00} & * & * \\  
     \widehat C_{10}^T & \ A_1 A_1^T + \widehat C_{11} & * \\ 
     \widehat C_{20} & \ A_2 A_1^T + \widehat C_{21} & A_2 A_2^T + \widehat C_{22}
  \end{array}\right) 
$
}


% Step 8: Insert the updates required to change the 
%         state from that given in Step 6 to that given in Step 7

\renewcommand{\update}{
$
  \begin{array}{l}          % do not delete this line 
    C_{11} = A_1 A_1^T + C_{11} \\
    C_{21} = A_2 A_1^T + C_{21} \\
  \end{array}               % do not delete this line 
$
}




\begin{figure}
	\begin{center}
		\FlaWorksheet
	\end{center}
	\caption{Blocked Algorithm for Loop Invariant 3}
	\label{fig:blk_inv3}
\end{figure}